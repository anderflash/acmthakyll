\documentclass[brazil,]{report}
\usepackage[xindy]{glossaries}
\usepackage{lmodern}
\usepackage{amssymb,amsmath}
\usepackage{ifxetex,ifluatex}
\usepackage{fixltx2e} % provides \textsubscript
% use upquote if available, for straight quotes in verbatim environments
\IfFileExists{upquote.sty}{\usepackage{upquote}}{}
\ifnum 0\ifxetex 1\fi\ifluatex 1\fi=0 % if pdftex
  \usepackage[utf8]{inputenc}
\else % if luatex or xelatex
  \ifxetex
    \usepackage{mathspec}
    \usepackage{xltxtra,xunicode}
  \else
    \usepackage{fontspec}
  \fi
  \defaultfontfeatures{Mapping=tex-text,Scale=MatchLowercase}
  \newcommand{\euro}{€}
\fi
% use microtype if available
\IfFileExists{microtype.sty}{\usepackage{microtype}}{}
\usepackage{longtable}
\ifxetex
  \usepackage[setpagesize=false, % page size defined by xetex
              unicode=false, % unicode breaks when used with xetex
              xetex]{hyperref}
\else
  \usepackage[unicode=true]{hyperref}
\fi
\hypersetup{breaklinks=true,
            bookmarks=true,
            pdfauthor={Anderson C. M. Tavares},
            pdftitle={Métodos para Detecção de Fixações, Sacadas e Outros Movimentos do Olhar},
            colorlinks=true,
            citecolor=blue,
            urlcolor=blue,
            linkcolor=magenta,
            pdfborder={0 0 0}}
\urlstyle{same}  % don't use monospace font for urls
\setlength{\parindent}{0pt}
\setlength{\parskip}{6pt plus 2pt minus 1pt}
\setlength{\emergencystretch}{3em}  % prevent overfull lines
\setcounter{secnumdepth}{5}
\ifxetex
  \usepackage{polyglossia}
  \setmainlanguage{brazil}
  \setotherlanguage{english}
\else
  \usepackage[brazil]{babel}
\fi

\title{Métodos para Detecção de Fixações, Sacadas e Outros Movimentos do Olhar}
\author{Anderson C. M. Tavares}
\date{Universidade de São Paulo}

\usepackage{titlesec}
\newcommand{\sectionbreak}{\clearpage}

\newacronym{I-VT} {I-VT}{\emph{Identification by Velocity Threshold}}
\newacronym{I-DT} {I-DT}{\emph{Identification by Velocity Threshold}}
\newacronym{I-VVT}{I-VVT}{\emph{Identification by Velocity and Velocity Threshold}}
\newacronym{I-VDT}{I-VDT}{\emph{Identification by Velocity and Dispersion Threshold}}
\newacronym{I-VMP}{I-VMP}{\emph{Identification by Velocity and Movement Pattern}}
\newacronym{I-HMM}{I-HMM}{\emph{Identification by Hidden Markov Model}}
\newacronym{I-MST}{I-MST}{\emph{Identification by Minimum Spanning Tree}}
\newacronym{I-AOI}{I-AOI}{\emph{Identification by Area Of Interest}}
\newacronym{I-AAT}{I-AAT}{\emph{Identification by Acceleration Adaptive-Threshold}}
\newacronym{FIR}{FIR}{\emph{Finite Impulse Response}}

\renewcommand*{\glossaryname}{Lista de Termos}
%\deftranslation{Glossary}{Lista de Termos}

\makeglossaries

\begin{document}
\maketitle

\begin{english}
\begin{abstract}
Complex systems imply to complex interfaces for users. With the demand with more accessible and effective interfaces, eye tracking is developing in a fast way. Eye movements can reflect user's cognitive processes and desires. To know these movements, several algorithms were and are being proposed for eye data classification basically in fixations and saccades. This work lists some known algorithms, their characteristics, advantages, disadvantages and applications. Works that analyzed some of these methods also are cited. According to the authors, there is no best classification method and also there is no best evaluation method of classification algorithms.
\end{abstract}
\end{english}
\newpage
\selectlanguage{brazil}
\begin{abstract}
Sistemas complexos implicam em interfaces complexas para os usuários. Na demanda por interfaces mais acessíveis e eficazes, a área de rastreamento do olhar vem se desenvolvendo rapidamente. Movimentos do olhar podem refletir processos cognitivos e desejos. Para conhecer esses movimentos, vários algoritmos foram e estão sendo propostos para classificação dos dados do olhar basicamente em fixações e sacadas. Este trabalho lista alguns algoritmos conhecidos, suas características, vantagens, desvantagens e aplicações. Trabalhos que analisaram alguns desses métodos também são citados. De acordo com os autores, não há melhor algoritmo de classificação e também não há um melhor método de avaliação do algoritmos de classificação.
\end{abstract}

{
\hypersetup{linkcolor=black}
\setcounter{tocdepth}{3}
\tableofcontents
\newpage
}

\printglossary[type=\acronymtype]

\chapter{Introdução}\label{introduuxe7uxe3o}

Rastreamento do olhar é uma área que estuda o comportamento do olhar,
analisando o processamento visual e cognitivo da pessoa. Os movimentos
do olhar são coletados por equipamentos denominados rastreadores de
olhar (\emph{eye trackers}) para análise, comparação e para desenvolver
sistemas mais interativos.

Dependendo da taxa de amostragem do equipamento e da aplicação
pretendida, a quantidade de dados coletados poder ser muito grande para
sua análise. Nesse caso é necessária uma transformação destes dados em
informações do processamento visual e cognitivo durante o experimento.
Tal transformação é denominada identificação ou classificação dos
movimentos do olhar.

Os rastreadores basicamente informam a coordenada bidimensional de cada
ponto coletado e seu \emph{timestamp}. Outras propriedades também podem
ser informadas, como o diâmetro da pupila. As amostras configuram uma
série temporal. Os dados de tempo e espaço podem ser usados para
analisar os movimentos.

O objetivo deste trabalho é levantar uma lista de métodos de
identificação de fixações e sacadas. Outros movimentos do olhar que
sejam também classificados por esses métodos também serão citados.

A seção \hyperref[movimentos]{Movimentos} define os movimentos do olhar.
A seção \hyperref[filtros]{Filtros} mostra os filtros utilizados como
pré-processamento da classificação dos movimentos. A seção
\hyperref[muxe9todos]{Métodos} descrevem os algoritmos estudados. A
seção \hyperref[avaliauxe7uxe3o-dos-muxe9todos]{Avaliação dos Métodos}
mostra algumas métricas e técnicas de avaliação criados para comparar e
validar resultados dos algoritmos de identificação. A seção
\hyperref[conclusuxe3o]{Conclusão} finaliza o texto e levanta tarefas e
trabalhos futuros.

\hyperdef{}{movimentos}{\chapter{Movimentos}\label{movimentos}}

Os movimentos do olhar podem ser:

\begin{itemize}
\itemsep1pt\parskip0pt\parsep0pt
\item
  \textbf{Fixações}: Para Salvucci {[}1{]} e Karrsgard{[}2{]}, uma
  fixação é uma pausa sobre regiões de interesse.
\item
  \textbf{Sacadas}: Um rápido movimento entre fixações {[}1{]};
\item
  \textbf{Movimentos fixacionais}: O olho não fica parado durante uma
  fixação, todavia ocorrem pequenos movimentos, como \emph{drifts}
  (desvios do foco para fora do alvo durante a fixação), tremores
  (rápidos, de baixa amplitude e involuntários), microssacadas
  (correções dos \emph{drifts} e renovação do estímulo na retina) e
  nystagmus (patologia periódica composta por movimentos suaves e
  rápidos alternadamente, que pode provocar tontura e sensação de
  movimento em objetos estáticos).
\item
  \textbf{Perseguição Contínua}: O movimento do olho acompanhando um
  alvo em movimento é denominado \emph{perseguição contínua},
  \emph{perseguição suave}, ou simplesmente \emph{perseguição}.
\item
  \textbf{Glissadas}: No fim da sacada, geralmente o olho não para no
  ponto desejado, e sim ele o passa. Dessa forma, ele faz um pequeno
  movimento senoidal para corrigir e fixar-se no ponto. Esse movimento é
  denominado \emph{glissada}.
\end{itemize}

\hyperdef{}{filtros}{\chapter{Filtros}\label{filtros}}

A análise de dados do olhar é dividida em duas partes: filtragem e
classificação. Os ruídos também são divididos em duas categorias:

\begin{itemize}
\itemsep1pt\parskip0pt\parsep0pt
\item
  Ruídos provenientes do equipamento;
\item
  Movimentos do olhar que não estão sendo analisados.
\end{itemize}

Os filtros servem para remover, ou pelo menos reduzir, o primeiro tipo
de ruído. A classificação elimina o segundo tipo. Essa separação em duas
etapas faz com que os métodos de classificação sejam mais independentes
dos equipamentos e suas características.

Alguns filtros utilizandos na literatura são:

\begin{itemize}
\itemsep1pt\parskip0pt\parsep0pt
\item
  Filtro \gls{FIR} {[}3{]};
\item
  Filtro de Média {[}4{]};
\item
  Filtro de Mediana {[}4{]};
\item
  Filtro Savitzky-Golay {[}5{]};
\item
  Filtro de Kalman {[}6{]}.
\end{itemize}

Filtros de média e mediana permitem reduzir o nível de ruído. Eles são
usados pelo sistema da Tobii {[}4{]}. O filtro da média aumenta a
duração da sacada, reduzindo sua velocidade e reduzindo a duração das
fixações podendo comprometer as identificações, além de criar falsos
pontos que não foram detectados pelo equipamento. O filtro da mediana
não cria falsos pontos, o resultado do filtro são pontos existentes das
amostras coletadas, não interfere na amplitude da sacada e na sua
duração (mantendo a mudança brusca de posições durante a sacada).

\renewcommand{\sectionbreak}{}

\section{Filtro de Resposta ao Impulso
Finita}\label{filtro-de-resposta-ao-impulso-finita}

Os filtros \gls{FIR} principais são:

\begin{itemize}
\itemsep1pt\parskip0pt\parsep0pt
\item
  Filtro Passa-Baixa;
\item
  Filtro Passa-Banda (ou Passa-Faixa);
\item
  Filtro Passa-Alta.
\end{itemize}

O filtro Passa-Baixa pode ser usado para atenuar as altas frequências do
ruído em um sinal de dados do olhar, além de poder ser desenvolvido em
\emph{hardware} utilizando um circuito RC (Resistor-Capacitor){[}3{]}. O
sinal da sacada pode ficar comprometido devido a resposta não ideal do
filtro. O resultado pode ser melhorado com um filtro FIR dinâmico
{[}3{]}.

\section{Savitzky-Golay}\label{savitzky-golay}

Nyström {[}5{]} argumenta sobre o uso do filtro Savitzky-Golay que:

\begin{itemize}
\itemsep1pt\parskip0pt\parsep0pt
\item
  O filtro não faz nenhuma forte premissa sobre as curvas das
  velocidades dos movimentos;
\item
  Mantém boa qualidade de informação sobre dados em alta frequência;
\item
  Mantém os máximos e mínimos locais (espacial e temporal);
\end{itemize}

A descrição em alto nível do processo de filtragem consiste em:

\begin{itemize}
\itemsep1pt\parskip0pt\parsep0pt
\item
  Achara a função polinomial que melhor descreve os dados;
\item
  Derivar analiticamente a função (para não adicionar ruído);
\item
  Fazer uma amostragem das funções usando a frequência original.
\end{itemize}

\section{Filtro de Kalman}\label{filtro-de-kalman}

O filtro de Kalman permite modelar sistemas cujos sinais sejam ruidosos,
mas que tenham um comportamento dinâmico observado e conhecido. Sauter
et al. {[}6{]} utiliza o filtro de Kalman para detectar perseguições,
mesmo que o sinal tenha ruído e sacadas corretivas que mantém o algo
novamente ao foco durante a perseguição. Ele realiza testes Chi-Quadrado
para verificar se a hipótese nula (perseguição) se mantém devido aos
valores estatísticos (média e desvio padrão) do sinal, ou se os novos
dados são de uma distribuição diferente (hipótese da sacada).

\chapter{Taxonomia de Métodos de
Classificação}\label{taxonomia-de-muxe9todos-de-classificauxe7uxe3o}

Salvucci {[}1{]} introduz uma taxonomia de algoritmos de identificação
de fixações e sacadas. Esta taxonomia é baseada em como são usadas as
informações de tempo e espaço. Os algoritmos citados no artigo
representam classes de técnicas que compartilham algum critério de
identificação. Ele também apresenta uma forma de analisar os algoritmos
de maneira qualitativa: facilidade de uso, velocidade de interpretação,
acurácia, robustez, e parametrização.

\section{Critérios espaciais}\label{crituxe9rios-espaciais}

\begin{itemize}
\itemsep1pt\parskip0pt\parsep0pt
\item
  Baseados em velocidade: Estes algoritmos utilizam o fato dos pontos
  que compõem uma fixação terem uma velocidade baixa, enquanto que numa
  sacada, as velocidades dos pontos são altas. Geralmente são utilizadas
  para classificar sacadas;
\item
  Baseados em dispersão: Em uma fixação, os pontos são próximos entre
  si. Medidas de dispersão podem ser utilizadas para classificar
  fixações;
\item
  Baseados em área de interesse (AOI): Os pontos são agrupados de acordo
  com regiões pré-determinadas pela aplicação. Servem geralmente para
  uma análise de alto nível. Podem ser usados algoritmos baseados em
  velocidade e/ou dispersão antes de realizar a análise baseado em AOI.
\end{itemize}

\section{Critérios temporaais}\label{crituxe9rios-temporaais}

\begin{itemize}
\itemsep1pt\parskip0pt\parsep0pt
\item
  Sensível à duração: Utiliza uma duração mínima para descartar fixações
  com uma duração curta demais para os limites fisiológicos do olho.
\item
  Localmente adaptativo: pontos vizinhos influenciam a classificação de
  um ponto específico. Robusto contra ruídos.
\end{itemize}

\renewcommand{\sectionbreak}{\clearpage}

\hyperdef{}{muxe9todos}{\chapter{Métodos}\label{muxe9todos}}

Os métodos a seguir classificam os dados do olhar coletados pelo
equipamento durante o experimento em sacadas, fixações, perseguições e
outros movimentos. Nem todos os algoritmos são desenvolvidos para
detectar todos os movimentos.

\begin{longtable}[c]{@{}lllll@{}}
\hline\noalign{\medskip}
Nome & Fixação & Sacada & Perseguição & Glissadas
\\\noalign{\medskip}
\hline\noalign{\medskip}
I-VT & X & X & &
\\\noalign{\medskip}
I-HMM & X & X & &
\\\noalign{\medskip}
I-DT & X & O que não & &
\\\noalign{\medskip}
& & for fixação & &
\\\noalign{\medskip}
I-MST & X & X & &
\\\noalign{\medskip}
I-AOI & Pontos dentro & Pontos fora & &
\\\noalign{\medskip}
& das áreas & das áreas & &
\\\noalign{\medskip}
I-VDT-Komogortsev & X & & &
\\\noalign{\medskip}
I-VT-Nyström & X & X & & X
\\\noalign{\medskip}
I-HMM-Karrsgard & X & X & &
\\\noalign{\medskip}
I-VVT & X & X & X &
\\\noalign{\medskip}
I-VMP & X & X & X &
\\\noalign{\medskip}
I-VDT & X & X & X &
\\\noalign{\medskip}
I-AAT & X & X & & X
\\\noalign{\medskip}
Kalman & & X & X &
\\\noalign{\medskip}
Clusterização & X & X & &
\\\noalign{\medskip}
\hline
\end{longtable}

Os algoritmos proposto por Salvucci representam as características mais
básicas e distintas dos algoritmos criados e publicados antes do seu
artigo de revisão: \gls{I-VT}, \gls{I-DT}, \gls{I-HMM}, \gls{I-MST},
\gls{I-AOI}.

Partindo destes métodos, melhorias e outros métodos foram propostos por
outros pesquisadores, como Komogortsev{[}Komogortsev2013{]},
Behrens{[}7{]}, Nyström{[}5{]}, Karrsgard{[}2{]} e Urruty.

\section{\gls{I-VT}}\label{section}

Este algoritmo representativo proposto por Salvucci é um dos mais
básicos. Contém um parâmetro, o limiar de velocidade. Recebendo as
amostras dos pontos, calculam-se suas velocidades. Se a velocidade for
menor que o limiar, o respectivo ponto é classificado como fixação,
senão é classificado como sacada.

\subsection{Vantagens}\label{vantagens}

\begin{itemize}
\itemsep1pt\parskip0pt\parsep0pt
\item
  fácil de implementar;
\item
  eficiente;
\item
  pode ser executado em tempo real.
\end{itemize}

\subsection{Desvantagens}\label{desvantagens}

\begin{itemize}
\itemsep1pt\parskip0pt\parsep0pt
\item
  instável em pontos com velocidade próxima do threshold (precisa lidar
  com o ruído do equipamento e movimentos do olhar irrelevantes para a
  pesquisa);
\item
  Pode provocar alternâncias entre classificações, implicando em
  fixações e sacadas com poucos pontos, aumentando o número de fixações
  excluídas pelo critério de duração mínima;
\item
  Não é robusto;
\item
  Perseguições podem ser classificados como fixações ou sacadas
  dependendo de sua velocidade.
\end{itemize}

\section{\gls{I-HMM}}\label{section-1}

Este algoritmo utiliza uma máquina de 2 estados para classificar fixação
e sacada, recebendo parâmetros de distribuição das velocidades (média e
desvio padrão para cada estado), além das probabilidades de transição
entre estados. O modelo pode ser treinado para reestimar os parâmetros.

O modelo recebe 8 parâmetros:

\begin{itemize}
\itemsep1pt\parskip0pt\parsep0pt
\item
  A média e desvio padrão das velocidades de cada estado, totalizando 4;
\item
  As probabilidades de transição (de fixação/sacada para
  fixação/sacada), totalizando 4;
\end{itemize}

Os parâmetros podem ser reestimados em um treinamento inicial.

\subsection{Vantagens:}\label{vantagens-1}

\begin{itemize}
\itemsep1pt\parskip0pt\parsep0pt
\item
  Modelo probabilístico ao invés de um threshold. Utiliza informação
  sequencial (os vizinhos influenciam o ponto);
\item
  É mais robusto na presença do ruído;
\item
  Pode expandir o diagrama de estados (incorporando mais movimentos do
  olhar);
\item
  É executado em tempo linear e pode ser executado em tempo real;
\item
  Larsson {[}8{]} não encontrou melhora significativa entre o
  \gls{I-HMM} e o \gls{I-VT}.
\end{itemize}

\subsection{Desvantagens:}\label{desvantagens-1}

\begin{itemize}
\itemsep1pt\parskip0pt\parsep0pt
\item
  Mais complexo que \gls{I-VT};
\item
  Procedimento de reestimar os parâmetros também é complexo.
\end{itemize}

\section{\gls{I-DT}}\label{section-2}

Este algoritmo utiliza o critério de dispersão para agrupar os pontos em
uma fixação. Ele inicia uma janela com tamanho de acordo com a duração
mínima de uma fixação (fixações curtas são descartadas), geralmente 100
ms. Caso a medida de dispersão dos pontos dentro da janela for menor que
um limiar, então a janela é expandida até que a dispersão seja maior,
agrupando todos os pontos na janela como uma fixação.

Salvucci utilizou como critério de dispersão
$(Max_x - Min_x) + (Max_y - Min_y)$. Outras medidas de dispersão podem
ser usadas:

\begin{itemize}
\itemsep1pt\parskip0pt\parsep0pt
\item
  Distância entre qualquer um dos pontos;
\item
  Distância entre os pontos e o centro da fixação;
\item
  Desvio padrão das coordenadas.
\end{itemize}

\subsection{Vantagens:}\label{vantagens-2}

\begin{itemize}
\itemsep1pt\parskip0pt\parsep0pt
\item
  Algoritmo simples
\item
  Tempo linear;
\item
  Pode ser feito em tempo real;
\item
  Resultado parecido com a saída do \gls{I-HMM} (sendo mais robusto do
  que o \gls{I-VT}).
\end{itemize}

\subsection{Desvantagens:}\label{desvantagens-2}

\begin{itemize}
\itemsep1pt\parskip0pt\parsep0pt
\item
  Parâmetros interdependentes (ex: duração mínima alta e limiar de
  dispersão baixa pode não classificar nenhuma fixação);
\item
  Sensível a ruído no critério espacial (pode ultrapassar o limiar);
\item
  Possíveis fixações dispersas podem não ser classificadas.
\end{itemize}

\section{\gls{I-MST}}\label{section-3}

Este algoritmo cria uma estrutura de árvore que interliga os pontos de
tal forma que a soma dos comprimentos das arestas da árvore seja o menor
possível. Para construir a árvore, utiliza-se o algoritmo de Prim . É
localmente adaptativo por interligar os pontos aos seus vizinhos, direta
ou indiretamente.

\begin{itemize}
\itemsep1pt\parskip0pt\parsep0pt
\item
  Arestas maiores do que um limiar são consideradas sacadas.
\item
  Calcula-se a média e o desvio padrão para arestas locais;
\item
  Se o comprimento da aresta for maior do que a média + desvio, então a
  aresta é considerada sacada.
\item
  Identificam-se fixações pelos \emph{clusters} não divididos pelas
  sacadas.
\end{itemize}

\subsection{Vantagens:}\label{vantagens-3}

\begin{itemize}
\itemsep1pt\parskip0pt\parsep0pt
\item
  Robusto (pode usar variância e média para lidar com ruído);
\item
  Cria clusters de fixações
\item
  Podem-se usar outras caracterizações que não sejam meramente espaciais
  para classificar as fixações.
\end{itemize}

\subsection{Desvantagens:}\label{desvantagens-3}

\begin{itemize}
\itemsep1pt\parskip0pt\parsep0pt
\item
  Lento (tempo de execução exponencial);
\item
  Para cada ponto adicionado, é necessário achar o ponto mais próximo
  dentre vários para restruturar o cluster e separar os clusters.
\end{itemize}

\section{I-AOI}\label{i-aoi}

Este método de classificação de alto nível converte as amostras em
regiões de acordo com divisões da região do estímulo. São fixações os
pontos que estiverem dentro das regiões. Pontos fora das regiões são
considerados como sacadas.

Cada região é identificada com um símbolo. O resultado do método
transforma as amostras em uma sequência de símbolos, cujas sequências
podem ser comparadas entre si usando o algoritmo de Levenshtein.

\subsection{Vantagens:}\label{vantagens-4}

\begin{itemize}
\itemsep1pt\parskip0pt\parsep0pt
\item
  Tempo real (execução online);
\item
  Simples de implementar;
\item
  Análise de alto nível.
\end{itemize}

\subsection{Desvantagens:}\label{desvantagens-4}

\begin{itemize}
\itemsep1pt\parskip0pt\parsep0pt
\item
  Não lida bem com sacadas (incluídas nas fixações se estiverem dentro
  das regiões), aumentando a duração da fixação;
\item
  Longas sacadas são consideradas fixações nas regiões intermediárias;
\item
  Depende da aplicação (distribuição das regiões).
\end{itemize}

\section{I-VDT de Komogortsev}\label{i-vdt-de-komogortsev}

O algoritmo \gls{I-VDT} de Komogortsev{[}9{]} classifica fixações,
sacadas e perseguições contínuas. Argumenta-se que não há algoritmos
robustos que classifiquem esses movimentos ternários.

Ele apresenta e compara, além de seu artigo, outros dois algoritmos de
identificação de fixações, sacadas e perseguições contínuas: \gls{I-VVT}
(semelhante ao \gls{I-VT} de Salvucci {[}1{]} com dois limiares de
velocidade para as três classes), e \gls{I-VMP} de Agustin {[}10{]} e
Larsson{[}11{]}.

No I-VMP, dentro de uma janela temporal, calculam-se ângulos dos pontos
com o eixo horizontal, plotam-se esses ângulos em uma circunferência
unitária, calcula-se a centróide e sua distância com a coordenada (0,0)
da circunferência. Se essa distância for maior que um limiar, seus
pontos são considerados parte de uma perseguição contínua.

O algoritmo \gls{I-VDT} usa o limiar de velocidade (recebido como
parâmetro) para classificar as sacadas, e uma medida de dispersão para
classificar perseguições e fixações. A implementação dos algoritmos
estão disponíveis em {[}12{]}.

\subsection{Vantagens}\label{vantagens-5}

\begin{itemize}
\itemsep1pt\parskip0pt\parsep0pt
\item
  O \gls{I-VDT} é basicamente uma combinação do \gls{I-VT} com o
  \gls{I-DT} do Salvucci {[}1{]}, dessa forma sendo facilmente absorvido
  e rapidamente implementado;
\item
  Realiza uma classificação ternária. Primeira separa as sacadas pelo
  critério de velocidade, depois separa fixações das perseguições pelo
  critério da dispersão;
\item
  Pode ser executado em tempo real;
\item
  Pode-se alterar o cálculo de dispersão que melhor classifique os dados
  do experimento. Komogortsev utilizou a medida de dispersão de
  Salvucci.
\end{itemize}

\subsection{Desvantagens}\label{desvantagens-5}

\begin{itemize}
\itemsep1pt\parskip0pt\parsep0pt
\item
  Tem as mesmas desvantagens do \gls{I-VT} no que diz respeito a
  movimentos cujas velocidades estejam próximas ao limiar, podendo
  alternar entre sacadas e outros movimentos.
\item
  Tem as mesmas desvantagens do \gls{I-DT} no que diz respeito ao ruído.
\item
  Não é adaptativo, o limiar precisa ser calculado externamente e
  passado como um parâmetro.
\end{itemize}

\section{\gls{I-VT} Adaptativo de
Nyström}\label{adaptativo-de-nystruxf6m}

O algoritmo \gls{I-VT} de Nyström realiza um método de classificação de
sacadas, fixações e glissadas. Ele verificou que a duração das glissadas
são longas o suficiente para influenciar as interpretações das fixações
e sacadas, sendo importante preparar o algoritmo para tal movimento. Os
passos para a classificação que o método executa são:

\begin{itemize}
\itemsep1pt\parskip0pt\parsep0pt
\item
  Aplica-se o filtro Savitzky-Golay nos pontos;
\item
  Calculam-se as velocidades e acelerações ponto-a-ponto;
\item
  Captura-se uma janela de velocidades;
\item
  Obtém-se os picos de velocidade maiores que um limiar $PT_n$;
\item
  Para cada pico, calcula-se o início (\emph{onset}) e o fim
  (\emph{offset}) da sacada.
\item
  Obtém-se as glissadas cujo onset seja igual ao offset da sacada
  predecessora e seu offset seja o primeiro vale da primeira curva após
  a sacada.
\item
  As fixações são os pontos não classificados como sacadas ou glissadas.
\end{itemize}

Os limiares não são fixos, e sim calculados iterativamente utilizando os
níveis de ruído:

\begin{itemize}
\itemsep1pt\parskip0pt\parsep0pt
\item
  Um limiar inicial é escolhido, $PT_1$;
\item
  Para todas as velocidades abaixo de $PT_1$, calcula-se a média $\mu_1$
  e desvio padrão $\sigma_1$;
\item
  Atualiza-se o limiar $PT_n = \mu_{n-1} + 6\sigma_{n-1}$
\item
  Quando $|PT_n-PT_{n-1}| < 1^\circ/s$, então o limiar final é obtido.
\end{itemize}

\subsection{Vantagens}\label{vantagens-6}

\begin{itemize}
\itemsep1pt\parskip0pt\parsep0pt
\item
  Classificação ternária: glissadas, fixações e sacadas.
\item
  Os histogramas da duração de fixação, duração da sacada, velocidade de
  pico da sacada e aceleração de pico da sacada tem um formato mais
  comportado, semelhante a uma distribuição normal;
\item
  O cálculo dos limiares são baseados nos dados, e por isso, mais
  robustos ao ruído;
\item
  Os limiares podem ser calculados separadamente para cada usuário ou
  experimento;
\item
  O pesquisador tem um maior controle sobre as glissadas, podendo
  adicioná-las às fixações ou às sacadas.
\end{itemize}

\subsection{Desvantagens}\label{desvantagens-6}

\begin{itemize}
\itemsep1pt\parskip0pt\parsep0pt
\item
  Embora ele argumenta que a aceleração amplia o nível do ruído,
  obrigando o pesquisador utilizar uma filtragem mais robusta, podem
  haver métodos que melhorem o resultado do algoritmo que use apenas a
  velocidade.
\item
  Usando apenas a velocidade, detecção de outros movimentos pode não ser
  possível.
\item
  Não se usa medidas de dispersão para tornar o algoritmo mais robusto.
  Ex: o que foi classificado como fixação, pode ter sido na verdade uma
  perseguição contínua.
\end{itemize}

\section{\gls{I-DT} e HMM de Kärrsgård}\label{e-hmm-de-kuxe4rrsguxe5rd}

Kärrsgård utiliza o algoritmo de identificação baseado em dispersão para
classificar os dados do olhar, e posteriormente um Modelo de Markov
Oculto na aplicação de digitação pelo olhar. Ele utiliza as fixações
para realizar a seleção dos caracteres, convertendo-os em sequências de
símbolos (palavras) para posterior interpretação do sistema.

O método pode ser considerado um híbrido usando dispersão (para
transformar os dados em fixações) e AOI para transformar fixações em
alvos (as teclas) para formar as palavras.

\subsection{Vantagens}\label{vantagens-7}

\begin{itemize}
\itemsep1pt\parskip0pt\parsep0pt
\item
  As vantagens do I-DT também se aplicam;
\item
  As vantagens do I-AOI de Salvucci se aplicam ao modelo apresentado por
  Kärrsgård;
\end{itemize}

\subsection{Desvantagens}\label{desvantagens-7}

\begin{itemize}
\itemsep1pt\parskip0pt\parsep0pt
\item
  Como a aplicação do algoritmo utilizou fixações, então o sistema de
  digitação é baseado em \emph{dwell time}. Alguns trabalhos utilizam
  sacadas (onset e offset{[}13{]}) para retirar a necessidade de fixação
  nas teclas.
\item
  Larsson {[}8{]} analisou alguns algoritmos e concluiu que \gls{I-HMM}
  não resulta em uma melhora significativa em relação ao simples
  \gls{I-VT}. Ele argumentou que talvez mais testes podem ser
  necessários, por causa da importância do modelo frente à adição de
  novos estados para detecção robusta de outros movimentos.
\item
  As desvantagens do método I-DT e I-AOI de Salvucci, também se aplicam.
\end{itemize}

\section{Clusterização de Projeção de
Urruty}\label{clusterizauxe7uxe3o-de-projeuxe7uxe3o-de-urruty}

O algoritmo de Urruty et al. encontra \emph{clusters} de pontos em um
espaço de dimensionalidade menor e seleciona a projeção que melhor ajuda
na identificação destes \emph{clusters}. O sistema de coordenadas para
os dados pode ser aleatoriamente escolhido. Com isso, reduz-se a chance
de obter clusters se sobrepondo. O método utiliza técnicas de projeção e
técnicas baseadas em densidade.

\begin{itemize}
\itemsep1pt\parskip0pt\parsep0pt
\item
  Projetam-se os pontos nos eixos $X$ e $Y$;
\item
  Criam-se dois histogramas, um para cada eixo;
\item
  Divide-se o espaço bidimensional em regiões de acordo com os
  histogramas;
\item
  As regiões de alta densidade são analisadas em um processo de
  expansão, a fim de capturar pontos externos e próximos que mantenham a
  densidade da região.
\item
  Análise da duração das fixações são realizadas para validar as
  regiões.
\item
  Pontos de um mesmo \emph{cluster} separados por mais de um período
  mínimo (recebido como parâmetro) são considerados de diferentes
  fixações, dividindo o \emph{cluster} em dois ou mais;
\item
  Os clusters divididos devem ter duração maior do que o período mínimo;
\item
  Pontos que não estiverem nas regiões (ou \emph{clusters} inválidos)
  são considerados como ruídos (ou movimento sacádicos).
\end{itemize}

\subsection{Vantagens}\label{vantagens-8}

\begin{itemize}
\itemsep1pt\parskip0pt\parsep0pt
\item
  Urruty et al. verificam, usando o método de avaliação \emph{Precision
  and Recall}, explanado na seção
  \hyperref[avaliauxe7uxe3o-dos-muxe9todos]{Avaliação dos Métodos}, que
  seu método detecta melhor as informações das fixações do que os
  algoritmos de \gls{I-MST}, \gls{I-DT} e \gls{I-VT};
\item
  O método também foi analisado usando vídeos como estímulos. Foi
  reforçado a possibilidade de seu uso na aplicação de sumarização de
  vídeos, combinando áreas de interesses, imagens chaves do vídeo, ou na
  aplicação de indexação dos vídeos.
\end{itemize}

\subsection{Desvantagens}\label{desvantagens-8}

\begin{itemize}
\itemsep1pt\parskip0pt\parsep0pt
\item
  O método perde mais sacadas do que os outros métodos comparados;
\item
  O método ainda não foi comparado com métodos com limiares adaptativos;
\item
  A qualidade do resultado do método também não é conhecida na presença
  de estímulos em movimento que gerariam perseguições contínuas.
\end{itemize}

\section{I-AAT de de Behrens}\label{i-aat-de-de-behrens}

O algoritmo \gls{I-AAT} de Behrens {[}7{]} usa as acelerações das
amostras para classificar movimentos sacádicos e não sacádicos. A
diferença desse para seu trabalho anterior {[}14{]} se dá no cálculo do
limiar não ser fixo, e sim adaptativo.

O cálculo utiliza o desvio padrão de uma janela de acelerações. O limiar
de aceleração $AT = N\times\sigma$, onde $N$ é um valor passado como
parâmetro, é atualizado ao mover a janela. No momento em que o valor da
aceleração de um ponto é maior do que o limiar, é constatado o início de
um movimento sacádico. Outras intersecções são detectadas (em uma
sacada, o sinal da aceleração resulta em duas curvas, uma positiva e uma
negativa). O limiar durante a sacada é mantido constante por um período
passado como parâmetro. Após o período, o limiar é modificado
linearmente até atingir $N\times\sigma$, onde volta a ser recalculado
iterativamente.

Para separar movimentos sacádicos principais das não principais (como
glissadas), o método utiliza também a posição, monotonicamente crescente
ou decrescente. Caso haja menos de 3 intersecções durante a análise da
aceleração e o sinal da posição das amostras da possível sacada tenha
terminado, então o movimento sacádico é considerado não principal. O
método utiliza uma função para controlar os estados destas variáveis.

O método utiliza um filtro de convolução para obter a segunda derivada
do sinal (aceleração) junto com um Filtro de Resposta Finito (FIR)
Passa-Baixa para atenuar as altas frequências, embora o autor admite o
aumento da magnitude do sinal do ruído remanescente.

\subsection{Vantagens}\label{vantagens-9}

\begin{itemize}
\itemsep1pt\parskip0pt\parsep0pt
\item
  O método utiliza um limiar adaptativo para se adequar ao ruído de
  diferentes equipamentos e de dados de diferentes usuários;
\item
  Utiliza aceleração para detectar sacadas;
\item
  Pode identificar artefatos no sinal de aceleração próximos ao limiar;
\item
  Sacadas curtas podem conter sinais de velocidade irregulares. A
  utilização de aceleração pode ser utilizada para a devida detecção.
\end{itemize}

\subsection{Desvantagens}\label{desvantagens-9}

\begin{itemize}
\itemsep1pt\parskip0pt\parsep0pt
\item
  Não utiliza velocidade para tornar a detecção das sacadas e fixações
  mais robusta;
\item
  Utiliza a posição, confiando no processo de atenuação do ruído pelo
  filtro;
\end{itemize}

\section{Filtro de Kalman}\label{filtro-de-kalman-1}

Os filtros de Kalman podem ser usados para detectar perseguições. As
sacadas corretivas dentro da perseguição podem ser considerados, junto
com o problemas na coleta, parte de um ruído branco gaussiano. O filtro
de Kalman trabalha com inovações dentro da previsão do modelo (em um
processo recursivo) e o compara com os dados coletados para decidir se é
uma perseguição ou uma sacada {[}6{]}. O filtro minimiza o erro entre o
estado do sistema previsto e o estado do sistema observado, mesmo na
presença de ruído, sendo interessante no uso de identificação de
movimentos do olhar {[}15{]}.

\subsection{Vantagens}\label{vantagens-10}

\begin{itemize}
\itemsep1pt\parskip0pt\parsep0pt
\item
  Permite detectar perseguições;
\item
  Não é necessária informação a priori da dinâmica da sacada;
\item
  Robusto a ruídos;
\item
  Koh et al.{[}15{]} verificaram que dados dos usuário que usam óculos
  ou lentes foram melhor identificados pelo filtro de Kalman do que pelo
  \gls{I-VT}.
\end{itemize}

\subsection{Desvantagens}\label{desvantagens-10}

\begin{itemize}
\itemsep1pt\parskip0pt\parsep0pt
\item
  É necessária informação a priori da dinâmica da perseguição;
\item
  O modelo pode não representar fielmente a realidade dos dados, podendo
  comprometer a identificação.
\item
  Os parâmetros para a identificação dependem da frequência do
  equipamento e da acurácia na calibração do usuário, tornando o
  algoritmo mais complexo de ser implementado.
\end{itemize}

\hyperdef{}{avaliauxe7uxe3o-dos-muxe9todos}{\chapter{Avaliação dos
Métodos}\label{avaliauxe7uxe3o-dos-muxe9todos}}

Shic{[}16{]} explora diferentes algoritmos de identificação de fixações
mostrando que suas interpretações podem ser diferentes, mesmo
trabalhando com os mesmos dados coletados. Ele analisa os seguintes
algoritmos baseados em dispersão:

\begin{itemize}
\itemsep1pt\parskip0pt\parsep0pt
\item
  Dispersão de Distância: a distância entre dois pontos quaisquer na
  fixação não pode superar um limiar. É executado em $O(n^2)$;
\item
  Centróide: os pontos de uma fixação não podem ser mais distantes do
  que um limiar para sua centróide. Pode construir uma versão em tempo
  real, computando apenas os novos pontos;
\item
  Posição-Variância: modela o grupo de pontos como uma distribuição
  gaussiana, e não podem ultrapassar um desvio padrão de distância;
\item
  \gls{I-DT} de Salvucci: a soma da máxima distância horizontal com a
  máxima distância vertical deve ser menor que um limiar.
\end{itemize}

Ele viu que o tempo de fixação médio segue um comportamento linear para
valores que correspondem aos limites fisiológicos da visão foveal
(desvio padrão da dispersão até $1̣^\circ$ e tempo mínimo de fixação até
200ms), mesmo que o número de fixações e o total de tempo gasto nas
fixações forem não lineares.

Salvucci {[}1{]} avalia os algoritmos de acordo com critérios
subjetivos:

\begin{itemize}
\itemsep1pt\parskip0pt\parsep0pt
\item
  Facilidade de implementação;
\item
  Acurácia;
\item
  Velocidade;
\item
  Robustez;
\item
  Número de parâmetros.
\end{itemize}

O único critério quantitativo é o número de parâmetros, visto que ele
definiu os outros critérios qualitativamente, embora possam ser criadas
métricas para torná-los objetivos. Também não há um valor final devido à
subjetividade, todavia também pode ser criado um cálculo usando e
agregando os critérios de forma quantitativa.

Larsson {[}8{]} em sua tese apresenta um método de avaliação denominado
\emph{Precision and Recall}. O método classifica a saída dos algoritmos
em 4 tipos baseados na \emph{predição} (a saída do algoritmo) e no
\emph{padrão de outro} (a correta classificação):

\begin{itemize}
\itemsep1pt\parskip0pt\parsep0pt
\item
  Padrão de ouro como \emph{Sim} e Predição como \emph{Sim}: Verdadeiro
  Positivo (VP);
\item
  Padrão de ouro como \emph{Sim} e Predição como \emph{Não}: Falso
  Positivo (FP);
\item
  Padrão de ouro como \emph{Não} e Predição como \emph{Sim}: Falso
  Negativo (FN);
\item
  Padrão de ouro como \emph{Não} e Predição como \emph{Não}: Verdadeiro
  Negativo (VP).
\end{itemize}

O objetivo da etapa \emph{Precision} é saber a razão entre os
verdadeiros positivos -- o algoritmo classificou corretamente como
\emph{Sim} -- e todos os classificados como \emph{Sim} pelo algoritmo,
mesmo os falso positivos. O objetivo da etapa \emph{Recall} é saber a
razão entre os verdadeiros positivos e todos que deveriam ser
classificados como \emph{Sim}, de acordo com o padrão de ouro.

Precision = \[\frac{VP}{VP+FP}\]

Recall = \[\frac{VP}{VP+FN}\]

Komogortsev criou métricas quantitativas e qualitativas para avaliação
dos algoritmos de identificação de fixações, sacadas {[}17{]} e
perseguições {[}9{]}. São relações entre dados pertencentes a uma classe
de movimento e estímulos considerados dessa classe.

A métrica quantitativa da fixação considera a relação entre as fixações
detectadas e as fixações do estímulo. O valor não chega a 100\% se o
estímulo do experimento levar em conta fixações e sacadas.

A métrica quantitativa da sacada compara o total das amplitudes das
sacadas detectadas com o total das amplitudes das sacadas do estímulo. O
valor pode estar acima de 100\% se fixações forem classificadadas como
sacadas ou se houver problemas na coleta de dados.

Há também a métrica qualitativa da fixação, usando o cálculo da soma das
distâncias entre as fixações detectadas e o centro do estímulo. A
métrica pode não dar um bom valor se o usuário não olhar exatamente para
o centro do estímulo.

\hyperdef{}{conclusuxe3o}{\chapter{Conclusão}\label{conclusuxe3o}}

Esta revisão serve para conhecer os métodos de classificação de dados do
olhar e métodos de avaliação dos algoritmos de análise. Com o estudo,
percebe-se que não existe um melhor método de classificação, pois os
resultados das análises dependem das definições adotadas dos movimentos,
qualidade na coleta de dados, frequência da coleta, nível de ruído,
características fisiológicas dos olhos de diferentes usuários, filtragem
adotada, critérios usados na identificação, movimentos escolhidos para
detecção, entre outras características.

Mesmo assim, o resultado dessas análises podem contribuir com a
interação dos usuários com os sistemas, a fim de aumentar a eficiência,
eficácia, reduzir a fadiga, melhorar a resposta do sistema e aumentar a
produtividade.

Como trabalho futuro, as etapas de filtragem, bem como outros métodos de
classificação, serão descritas e analisadas. Classificações de outros
movimentos do olhar também serão levadas em conta, como piscadas,
perseguições contínuas, microssacadas, tremores e nystagmus.

Alguns dos métodos que estão sendo estudados para complementar este
texto são:

\begin{itemize}
\itemsep1pt\parskip0pt\parsep0pt
\item
  Variância e Covariância de Veneri {[}18{]};
\item
  \emph{Mean Shift Procedure} de Santella {[}19{]}.
\end{itemize}

\chapter{Bibliografia}\label{bibliografia}

{[}1{]} D. D. Salvucci and J. H. Goldberg, ``Identifying fixations and
saccades in eye-tracking protocols,'' in \emph{Proceedings of the
symposium on Eye tracking research \& applications - ETRA '00}, 2000,
pp. 71--78.

{[}2{]} I. Kärrsgård and A. Lindholm, ``Eye movement tracking using
hidden Markov models,'' PhD thesis.

{[}3{]} P. Olsson, ``Real-time and Offline Filters for Eye Tracking,''
PhD thesis, KTH Electrical Engineering; KTH Electrical Engineering,
Stockholm, Sweden, 2007.

{[}4{]} A. Olsen, ``The Tobii I-VT Fixation Filter,'' 2012.

{[}5{]} M. Nyström and K. Holmqvist, ``An adaptive algorithm for
fixation, saccade, and glissade detection in eyetracking data.''
\emph{Behavior research methods}, vol. 42, no. 1, pp. 188--204, Feb.
2010.

{[}6{]} D. Sauter, B. J. Martin, N. Renzo, and C. Vomscheid, ``Analysis
of eye tracking movements using innovations generated by a Kalman
filter,'' \emph{Medical \& Biological Engineering \& Computing}, vol.
29, no. 1, pp. 63--69, Jan. 1991.

{[}7{]} F. Behrens, M. Mackeben, and W. Schröder-Preikschat, ``An
improved algorithm for automatic detection of saccades in eye movement
data and for calculating saccade parameters.'' \emph{Behavior research
methods}, vol. 42, no. 3, pp. 701--8, Aug. 2010.

{[}8{]} G. Larsson, ``Evaluation methodology of eye movement
classification algorithms,'' PhD thesis, Royal Institute of Technology,
2010.

{[}9{]} O. V. Komogortsev and A. Karpov, ``Automated classification and
scoring of smooth pursuit eye movements in the presence of fixations and
saccades.'' \emph{Behavior research methods}, vol. 45, no. 1, pp.
203--15, Mar. 2013.

{[}10{]} J. S. Agustin, ``Off-the-shelf gaze interaction,'' PhD thesis,
2010.

{[}11{]} L. Larsson, ``Event detection in eye-tracking data,'' PhD
thesis, Lund University, 2010.

{[}12{]} O. V. Komogortsev, ``Eye movement Classification
{[}Software{]}.'' 2013, {[}Online{]}, Available:
\url{http://cs.txstate.edu/~ok11/emd_offline.html}. {[}Accessed:
14-Dec-2013{]}.

{[}13{]} O. V. Komogortsev, Y. S. Ryu, D. H. Koh, and S. M. Gowda,
``Instantaneous saccade driven eye gaze interaction,'' in
\emph{Proceedings of the International Conference on Advances in
Computer Enterntainment Technology - ACE '09}, 2009, p. 140.

{[}14{]} F. Behrens and L. Weiss, ``An algorithm separating saccadic
from nonsaccadic eye movements automatically by use of the acceleration
signal,'' \emph{Vision Research}, vol. 32, no. 5, pp. 889--893, May
1992.

{[}15{]} D. H. Koh, S. A. Munikrishne Gowda, and O. V. Komogortsev,
``Input evaluation of an eye-gaze-guided interface,'' in
\emph{Proceedings of the 1st ACM SIGCHI symposium on Engineering
interactive computing systems - EICS '09}, 2009, p. 197.

{[}16{]} F. Shic, B. Scassellati, and K. Chawarska, ``The incomplete
fixation measure,'' in \emph{Proceedings of the 2008 symposium on Eye
tracking research \& applications - ETRA '08}, 2008, p. 111.

{[}17{]} O. V. Komogortsev, S. Jayarathna, D. H. Koh, and S. M. Gowda,
``Qualitative and quantitative scoring and evaluation of the eye
movement classification algorithms,'' in \emph{Proceedings of the 2010
Symposium on Eye-Tracking Research \& Applications - ETRA '10}, 2010, p.
65.

{[}18{]} G. Veneri, P. Piu, F. Rosini, P. Federighi, A. Federico, and A.
Rufa, ``Automatic eye fixations identification based on analysis of
variance and covariance,'' \emph{Pattern Recognition Letters}, vol. 32,
no. 13, pp. 1588--1593, Oct. 2011.

{[}19{]} A. Santella and D. DeCarlo, ``Robust clustering of eye movement
recordings for quantification of visual interest,'' in \emph{Proceedings
of the Eye tracking research \& applications symposium on Eye tracking
research \& applications - ETRA'2004}, 2004, pp. 27--34.

\end{document}
