\section{Introdução}\label{introduuxe7uxe3o}

Rastreamento do olhar é uma área que estuda o comportamento do olhar,
analisando o processamento visual e cognitivo da pessoa. Os movimentos
do olhar são coletados por equipamentos denominados rastreadores de
olhar (\emph{eye trackers}) para análise, comparação e para desenvolver
sistemas mais interativos.

Dependendo da taxa de amostragem do equipamento e da aplicação
pretendida, a quantidade de dados coletados poder ser muito grande para
sua análise. Nesse caso é necessária uma transformação destes dados em
eventos que informem melhor o processamento visual e cognitivo durante o
experimento.

Os rastreadores basicamente informam a coordenada bidimensional de cada
ponto coletado e seu \emph{timestamp}. Outras propriedades também podem
ser informadas, como o diâmetro da pupila. As amostras configuram uma
série temporal. Os dados de tempo e espaço podem ser usados para
analisar os eventos.

\section{Eventos}\label{eventos}

Os eventos do movimento do olhar podem ser:

\begin{itemize}
\itemsep1pt\parskip0pt\parsep0pt
\item
  \textbf{Fixações}: Para Salvucci {[}@Salvucci2000{]} e
  Karrsgard{[}@Karrsgard2003{]}, uma fixação é uma pausa sobre regiões
  de interesse.
\item
  \textbf{Sacadas}: Um rápido movimento entre fixações {[}@Salvucci2000,
  @Karrsgard2003{]};
\item
  \textbf{Movimentos fixacionais}: O olho não fica parado durante uma
  fixação, todavia ocorrem pequenos movimentos, como \emph{drifts}
  (desvios do foco para fora do alvo durante a fixação), tremores
  (rápidos, de baixa amplitude e involuntários), microssacadas
  (correções dos \emph{drifts} e renovação do estímulo na retina) e
  nystagmus (patologia periódica composta por movimentos suaves e
  rápidos alternadamente, que pode provocar tontura e sensação de
  movimento em objetos estáticos).
\item
  \textbf{Perseguição Contínua}: O movimento do olho acompanhando um
  alvo em movimento é denominado \emph{perseguição contínua},
  \emph{perseguição suave}, ou simplesmente \emph{perseguição}.
\item
  \textbf{Glissadas}: No fim da sacada, geralmente o olho não para no
  ponto desejado, e sim ele o passa. Dessa forma, ele faz um pequeno
  movimento senoidal para corrigir e fixar-se no ponto. Esse movimento é
  denominado \emph{glissada}.
\end{itemize}

\section{Filtros}\label{filtros}

A análise de dados do olhar é dividida em duas partes: filtragem e
classificação. Os ruídos também são divididos em duas categorias:

\begin{itemize}
\itemsep1pt\parskip0pt\parsep0pt
\item
  Ruídos provenientes do equipamento;
\item
  Movimentos do olhar que não estão sendo analisados.
\end{itemize}

Os filtros servem para remover, ou pelo menos reduzir, o primeiro tipo
de ruído. A classificação elimina o segundo tipo. Essa separação em duas
etapas faz com que os métodos de classificação sejam mais independentes
dos equipamentos e suas características.

Alguns filtros utilizandos na literatura são:

\begin{itemize}
\itemsep1pt\parskip0pt\parsep0pt
\item
  Filtro de Resposta ao Impulso Finita (FIR) {[}@Olsson2007{]};
\item
  Filtro de Média {[}@Olsen2012{]};
\item
  Filtro de Mediana {[}@Olsen2012{]};
\item
  Filtro Savitzky-Golay {[}@Nystrom2010{]};
\item
  Filtro de Kalman {[}@Sauter1991{]}.
\end{itemize}

\section{Categorias de Métodos de
Classificação}\label{categorias-de-muxe9todos-de-classificauxe7uxe3o}

Salvucci {[}@Salvucci2000{]} introduz uma taxonomia de algoritmos de
identificação de fixações e sacadas. Esta taxonomia é baseada em como
são usadas as informações de tempo e espaço. Os algoritmos citados no
artigo representam classes de técnicas que compartilham algum critério
de identificação. Ele também apresenta uma forma de analisar os
algoritmos de maneira qualitativa: facilidade de uso, velocidade de
interpretação, acurácia, robustez, e parametrização.

\subsubsection{Critérios espaciais}\label{crituxe9rios-espaciais}

\begin{itemize}
\itemsep1pt\parskip0pt\parsep0pt
\item
  Baseados em velocidade: Estes algoritmos utilizam o fato dos pontos
  que compõem uma fixação terem uma velocidade baixa, enquanto que numa
  sacada, as velocidades dos pontos são altas. Geralmente são utilizadas
  para classificar sacadas;
\item
  Baseados em dispersão: Em uma fixação, os pontos são próximos entre
  si. Medidas de dispersão podem ser utilizadas para classificar
  fixações;
\item
  Baseados em área de interesse (AOI): Os pontos são agrupados de acordo
  com regiões pré-determinadas pela aplicação. Servem geralmente para
  uma análise de alto nível. Podem ser usados algoritmos baseados em
  velocidade e/ou dispersão antes de realizar a análise baseado em AOI.
\end{itemize}

\subsubsection{Critérios temporaais}\label{crituxe9rios-temporaais}

\begin{itemize}
\itemsep1pt\parskip0pt\parsep0pt
\item
  Sensível à duração: Utiliza uma duração mínima para descartar fixações
  com uma duração curta demais para os limites fisiológicos do olho.
\item
  Localmente adaptativo: pontos vizinhos influenciam a classificação de
  um ponto específico. Robusto contra ruídos.
\end{itemize}

\section{Métodos}\label{muxe9todos}

\begin{longtable}[c]{@{}llll@{}}
\hline\noalign{\medskip}
Nome & Fixação & Sacada & Perseguição Lenta
\\\noalign{\medskip}
\hline\noalign{\medskip}
I-VT & & &
\\\noalign{\medskip}
I-HMM & & &
\\\noalign{\medskip}
I-DT & & &
\\\noalign{\medskip}
I-MST & & &
\\\noalign{\medskip}
I-AOI & & &
\\\noalign{\medskip}
\hline
\end{longtable}

I-VDT Komogortsev ------------------ -------------- ----------------
------------------

Os algoritmos proposto por Salvucci representam as características mais
básicas e distintas dos algoritmos criados e publicados antes do seu
artigo de revisão: I-VT, I-DT, I-HMM, I-MST, I-AOI.

\subsection{I-VT}\label{i-vt}

Este algoritmo representativo proposto por Salvucci é um dos mais
básicos. Contém um parâmetro, o limiar de velocidade. Recebendo as
amostras dos pontos, calculam-se suas velocidades. Se a velocidade for
menor que o limiar, o respectivo ponto é classificado como fixação,
senão é classificado como sacada.

\subsubsection{Vantagens}\label{vantagens}

\begin{itemize}
\itemsep1pt\parskip0pt\parsep0pt
\item
  fácil de implementar;
\item
  eficiente;
\item
  pode ser executado em tempo real.
\end{itemize}

\subsubsection{Desvantagens}\label{desvantagens}

\begin{itemize}
\itemsep1pt\parskip0pt\parsep0pt
\item
  instável em pontos com velocidade próxima do threshold (precisa lidar
  com o ruído do equipamento e movimentos do olhar irrelevantes para a
  pesquisa);
\item
  Pode provocar alternâncias entre classificações, implicando em
  fixações e sacadas com poucos pontos, aumentando o número de fixações
  excluídas pelo critério de duração mínima;
\item
  Não é robusto;
\item
  Perseguições podem ser classificados como fixações ou sacadas
  dependendo de sua velocidade.
\end{itemize}

\subsection{I-HMM}\label{i-hmm}

Este algoritmo utiliza uma máquina de 2 estados para classificar fixação
e sacada, recebendo parâmetros de distribuição das velocidades (média e
desvio padrão para cada estado), além das probabilidades de transição
entre estados. O modelo pode ser treinado para reestimar os parâmetros.

\subsubsection{Vantagens:}\label{vantagens-1}

\begin{itemize}
\itemsep1pt\parskip0pt\parsep0pt
\item
  Modelo probabilístico ao invés de um threshold. Utiliza informação
  sequencial (os vizinhos influenciam o ponto);
\item
  É mais robusto na presença do ruído;
\item
  Pode expandir o diagrama de estados (incorporando mais movimentos do
  olhar);
\item
  É executado em tempo linear e pode ser executado em tempo real.
\end{itemize}

\subsubsection{Desvantagens:}\label{desvantagens-1}

\begin{itemize}
\itemsep1pt\parskip0pt\parsep0pt
\item
  Mais complexo que I-VT;
\item
  Procedimento de reestimar os parâmetros também é complexo.
\end{itemize}

\subsection{I-DT}\label{i-dt}

Este algoritmo utiliza o critério de dispersão para agrupar os pontos em
uma fixação. Ele inicia uma janela com tamanho de acordo com a duração
mínima de uma fixação (fixações curtas são descartadas), geralmente 100
ms. Caso a medida de dispersão dos pontos dentro da janela for menor que
um limiar, então a janela é expandida até que a dispersão seja maior,
agrupando todos os pontos na janela como uma fixação. Salvucci utilizou
como critério de dispersão $(Max_x - Min_x) + (Max_y - Min_y)$. Outras
medidas de dispersão podem ser usadas:

\begin{itemize}
\itemsep1pt\parskip0pt\parsep0pt
\item
  Distância entre qualquer um dos pontos;
\item
  Distância entre os pontos e o centro da fixação;
\item
  Desvio padrão das coordenadas.
\end{itemize}

\subsubsection{Vantagens:}\label{vantagens-2}

\begin{itemize}
\itemsep1pt\parskip0pt\parsep0pt
\item
  Algoritmo simples
\item
  Tempo linear;
\item
  Pode ser feito em tempo real;
\item
  Resultado parecido com a saída do I-HMM (sendo mais robusto do que o
  I-VT).
\end{itemize}

\subsubsection{Desvantagens:}\label{desvantagens-2}

\begin{itemize}
\itemsep1pt\parskip0pt\parsep0pt
\item
  Parâmetros interdependentes (ex: duração mínima alta e limiar de
  dispersão baixa pode não classificar nenhuma fixação);
\item
  Sensível a ruído no critério espacial (pode ultrapassar o limiar);
\item
  Possíveis fixações dispersas podem não ser classificadas.
\end{itemize}

\subsection{I-MST}\label{i-mst}

Este algoritmo cria uma estrutura de árvore que interliga os pontos de
tal forma que a soma dos comprimentos das arestas da árvore seja o menor
possível. Para construir a árvore, utiliza-se o algoritmo de Prim
{[}@Prim1988{]}. É localmente adaptativo por interligar os pontos aos
seus vizinhos, direta ou indiretamente.

\subsubsection{Vantagens:}\label{vantagens-3}

\begin{itemize}
\itemsep1pt\parskip0pt\parsep0pt
\item
  Robusto (pode usar variância e média para lidar com ruído);
\item
  Cria clusters de fixações
\item
  Podem-se usar outras caracterizações que não sejam meramente espaciais
  para classificar as fixações.
\end{itemize}

\subsubsection{Desvantagens:}\label{desvantagens-3}

\begin{itemize}
\itemsep1pt\parskip0pt\parsep0pt
\item
  Lento (tempo de execução exponencial);
\item
  Para cada ponto adicionado, é necessário achar o ponto mais próximo
  dentre vários para restruturar o cluster e separar os clusters.
\end{itemize}

\subsection{I-AOI}\label{i-aoi}

Este método de classificação de alto nível converte as amostras em
regiões de acordo com divisões da região do estímulo. Cada região é
identificada com um símbolo. O resultado do método transforma as
amostras em uma sequência de símbolos, cujas sequências podem ser
comparadas entre si usando o algoritmo de
Levenshtein{[}@Cristino2010{]}.

\subsubsection{Vantagens:}\label{vantagens-4}

\begin{itemize}
\itemsep1pt\parskip0pt\parsep0pt
\item
  Tempo real;
\item
  Simples;
\item
  Análise de alto nível.
\end{itemize}

\subsubsection{Desvantagens:}\label{desvantagens-4}

\begin{itemize}
\itemsep1pt\parskip0pt\parsep0pt
\item
  Não lida bem com sacadas (incluídas nas fixações se estiverem dentro
  das regiões), aumentando a duração da fixação;
\item
  Longas sacadas são consideradas fixações nas regiões intermediárias;
\item
  Depende da aplicação (distribuição das regiões).
\end{itemize}

\section{I-VDT de Komogortsev}\label{i-vdt-de-komogortsev}

O algoritmo de Komogortsev{[}@Komogortsev2013{]} classifica fixações,
sacadas e perseguições contínuas. Argumenta-se que não há algoritmos
robustos que classifiquem esses movimentos ternários.

\subsection{Vantagens}\label{vantagens-5}

\subsection{Desvantagens}\label{desvantagens-5}

\section{I-VT Adaptativo de
Nyström}\label{i-vt-adaptativo-de-nystruxf6m}

\subsection{Vantagens}\label{vantagens-6}

\subsection{Desvantagens}\label{desvantagens-6}

\section{I-HMM de Karrsgard}\label{i-hmm-de-karrsgard}

\subsection{Vantagens}\label{vantagens-7}

\subsection{Desvantagens}\label{desvantagens-7}

\section{Clusterização de Projeção de
Urruty}\label{clusterizauxe7uxe3o-de-projeuxe7uxe3o-de-urruty}

\subsection{Vantagens}\label{vantagens-8}

\subsection{Desvantagens}\label{desvantagens-8}

\section{Avaliação dos Métodos}\label{avaliauxe7uxe3o-dos-muxe9todos}

Shic{[}@Shic2008{]} explora diferentes algoritmos de identificação de
fixações mostrando que suas interpretações podem ser diferentes, mesmo
trabalhando com os mesmos dados coletados. Ele analisa os seguintes
algoritmos baseados em dispersão:

\begin{itemize}
\itemsep1pt\parskip0pt\parsep0pt
\item
  Dispersão de Distância: a distância entre dois pontos quaisquer na
  fixação não pode superar um limiar. É executado em $O(n^2)$;
\item
  Centróide: os pontos de uma fixação não podem ser mais distantes do
  que um limiar para sua centróide. Pode construir uma versão em tempo
  real, computando apenas os novos pontos;
\item
  Posição-Variância: modela o grupo de pontos como uma distribuição
  gaussiana, e não podem ultrapassar um desvio padrão de distância;
\item
  I-DT de Salvucci: a soma da máxima distância horizontal com a máxima
  distância vertical deve ser menor que um limiar.
\end{itemize}

Ele viu que o tempo de fixação médio segue um comportamento linear para
valores que correspondem aos limites fisiológicos da visão foveal
(desvio padrão da dispersão até $1̣^\circ$ e tempo mínimo de fixação até
200ms), mesmo que o número de fixações e o total de tempo gasto nas
fixações forem não lineares.

Salvucci {[}@Salvucci2000{]} avalia os algoritmos de acordo com
critérios subjetivos:

\begin{itemize}
\itemsep1pt\parskip0pt\parsep0pt
\item
  Facilidade de implementação;
\item
  Acurácia;
\item
  Velocidade;
\item
  Robustez;
\item
  Número de parâmetros.
\end{itemize}

O único critério quantitativo é o número de parâmetros, visto que ele
definiu os outros critérios qualitativamente, embora possam ser criadas
métricas para torná-los objetivos. Também não há um valor final devido à
subjetividade, todavia também pode ser criado um cálculo usando e
agregando os critérios de forma quantitativa.

Larsson {[}@Larsson2010{]} em sua tese apresenta um método de avaliação
denominado \emph{Precision and Recall}. O método classifica a saída dos
algoritmos em 4 tipos baseados na \emph{predição} (a saída do algoritmo)
e no \emph{padrão de outro} (a correta classificação):

\begin{itemize}
\itemsep1pt\parskip0pt\parsep0pt
\item
  Padrão de ouro como \emph{Sim} e Predição como \emph{Sim}: Verdadeiro
  Positivo (VP);
\item
  Padrão de ouro como \emph{Sim} e Predição como \emph{Não}: Falso
  Positivo (FP);
\item
  Padrão de ouro como \emph{Não} e Predição como \emph{Sim}: Falso
  Negativo (FN);
\item
  Padrão de ouro como \emph{Não} e Predição como \emph{Não}: Verdadeiro
  Negativo (VP).
\end{itemize}

O objetivo da etapa \emph{Precision} é saber a razão entre os
verdadeiros positivos -- o algoritmo classificou corretamente como
\emph{Sim} -- e todos os classificados como \emph{Sim} pelo algoritmo,
mesmo os falso positivos. O objetivo da etapa \emph{Recall} é saber a
razão entre os verdadeiros positivos e todos que deveriam ser
classificados como \emph{Sim}, de acordo com o padrão de ouro.

Precision = \[\frac{VP}{VP+FP}\]

Recall = \[\frac{VP}{VP+FN}\]

\section{Conclusão}\label{conclusuxe3o}

Esta revisão serve para conhecer os métodos de classificação de dados do
olhar e métodos de avaliação dos algoritmos de análise. Como trabalho
futuro, as etapas de filtragem, bem como outros métodos de
classificação, serão descritas e analisadas. Classificações de outros
eventos do olhar também serão levadas em conta, como piscadas,
perseguições contínuas e nystagmus. Alguns dos métodos que estão sendo
avaliados são:

\begin{itemize}
\itemsep1pt\parskip0pt\parsep0pt
\item
  HMM de Karrsgard {[}@Karrsgard2003{]};
\item
  Clusterização de Projeção de Urruty {[}@Urruty2007a{]};
\item
  Variância e Covariância de Veneri {[}@Veneri2011{]};
\item
  \emph{Mean Shift Procedure} de Santella {[}@Santella2004{]}.
\end{itemize}

\section{Bibliografia}\label{bibliografia}
